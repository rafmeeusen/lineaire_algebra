\documentclass{article}
\usepackage{graphicx} % Required for inserting images
\usepackage{amsmath}
\usepackage{amssymb}

\title{Mijn samenvatting lineaire algebra}
\author{Raf Meeusen}
\date{2023-2024}

% mark paragraphs with empty line instead of indented first line
\setlength{\parindent}{0em}
\setlength{\parskip}{1em}

\begin{document}



\maketitle

% H1 
\section{Eerstegraadsvergelijkingen en matrices}

\subsection{Definities, notaties}

\begin{itemize}
    \item p.15-17: drie ERO's, type I ($\lambda R_i$), type II ($R_i \leftrightarrow R_j$), type III ($R_i \rightarrow R_i + \lambda R_j$)
    \item p.20: definitie 1.5 (leidend element, echolonform, trapvorm, rijgereduceerd) 
    \item p.21: definitie 1.6 (gebonden variabelen, vrije variabelen) 
    \item p.28-29: $A=(a_{ij})$ en $(A)_{ij} = a_{ij}$, soms met $1 \leq i \leq m$ etc. erbij
    \item p.29: "we werken analoog voor kolomvectoren" TODO ik snap dit niet goed; is $\mathbb{R}^n$ voor zowel rij- als kolomvectoren? 
    \item p.30 definitie 1.18  
    \item p.32 definitie 1.22 matrixproduct;  TODO eens zelf uitschrijven zonder te spieken
    \item p.32 definitie 1.23: scalaire matrix
    \item p.34 definitie $A^n$, en afspraak $A^0 = \mathbb{I}_n$
    \item p.34 definitie spoort (trace) 
    \item p.35 definitie 1.29 links inverse en rechts inverse van $A \in \mathbb{R}^{m \times n} $; matrix vermenigvuldiging dus dimensies moeten kloppen, en bij niet-vierkante heeft links-inverse andere dimensies dan rechts-inverse! 
    \item p.36 definitie 1.33: inverteerbare = reguliere = niet-singuliere matrix; omgekeerde: niet-inverteerbaar = singulier
    \item definitie 1.36: elementaire matrix van type I, II of III, notatie $E_n$
    \item definitie 1.43: bovendriehoeks en benedendriehoeks: ook voor niet-vierkante matrices! 
\end{itemize}


\subsection{Stellingen}

\begin{itemize}
    \item Propositie 1.8 (voor \emph{elke} matrix) 
    \item Stelling 1.11 (strijdig/1 oplossing/oneindig veel oplossingen) 
    \item Stelling 1.32 (p.36): hier al (eenvoudig) aangetoond dat bij vierkante $A$ de links en rechts inverse (als ze bestaan) gelijk moeten zijn 
    \item Stelling 1.38: goed begrijpen, onthouden
    \item Stelling 1.39: grappig bewijs in de zin dat equivalentie wordt bewezen in vorm van slang die zijn eigen staart bijt. 
    \item Lemma 1.44: product van $L_1$ en $L_2$ is opnieuw een $L$; idem voor $U$ ; $L$ benedendriehoeks, $U$ bovendriehoeks. 
\end{itemize}

\subsection{Redeneringen, inzichten}

p. 15 al: stelsel in matrixvorm: $A\cdot X=B$. Met in $X$ variabelen. Als $B=0$ dan homogeen stelsel. 

ERO's veranderen de oplossingsruimte van het stelsel niet. Voor type I en II redelijk triviaal, voor type III minder triviaal (p.17: kan je dit uitleggen). TODO: ooit als oefening gedaan denk ik. 
Elke ERO kan ongedaan gemaakt worden met inverse ERO, en telkens is inverse ERO van zelfde type. 

De verbanden tussen $m \times n$ matrix in echelonvorm, aantal niet-nul rijen $r$, en oplosbaarheid: TODO eens goed voor mezelf samenvatten. Vraag: kan $r > n$  zijn?? 

TODO: eens goed nakijken; zijn de type III operaties altijd met $j<i$ als we Gauss doen?? (zie verder LU decomp). 

p.31: eigenschappen optelling. Vooral $(A+B)^T = A^T + B^T =  B^T+A^T  $ onthouden en inzien. 

Stel $A =  \begin{pmatrix} a_1 \\ a_2 \\ ... \\ a_q  \end{pmatrix}$ met $a_i$ de rijen van $A$. En stel $B = (b_1 b_2 ... b_p)$, met $b_i$ kolommen van $B$. 
Dan $A \cdot B = \begin{pmatrix} A\cdot b_1 & A\cdot b_2 & ... & A\cdot b_p) \end{pmatrix}$  (zie p. 33 punt 2). 
En er is een rij-equivalent die hier niet bij staat: $A \cdot B = \begin{pmatrix} a_1\cdot B \\  a_2\cdot B \\ ... \\ a_q\cdot B \end{pmatrix}$. TODO: nakijken of dit allemaal klopt. 
Ook: $1 \times n$ maal $n \times 1$ geeft 1 bij 1, maar omgekeerd n bij n. En ook: $n \times 1$ maal $1 \times m$ geeft $n \times m$, niet vierkant dus. En $1 \times m$ maal $B = m \times n$ heeft $n \times m$. 
Soit, dit allemaal nog eens samenvatten misschien? 

p.33 eigenschappen nu we ook vermenigvuldiging hebben. $(A \cdot B)^T = A^T B^T$. En $A \cdot \mathbb{I}_n = \mathbb{I}_n \cdot A$. 

Noot: als we een matrix zien als een "operatie", en een vermenigvuldiging van matrices zien als de opeenvolging van die twee operaties, dan vinden we het ook logisch dat de uitkomst van $AB$ niet zelfde is als $BA$. (schoenen aantrekken, dan kousen, of omgekeerd. 

p.34 puntje 7: toch wel als $A$ inverteerbaar is? Komt normaal nog wel ergens terug... 

TODO: opdracht 1.25 (p.34) voor inzicht

TODO: opdracht 1.27 (p.34) voor inzicht

p. 36, eigenschap 1.31: als $A \times B = \mathbb{I}_m$ dan is  $B^T \times A^T = \mathbb{I}_m$. Evident maar interessant. 

TODO: opdracht 3 en 4 p. 36

Elementaire operatie met elementaire matrix: links vermenigvuldigen! $E_n \times A$. 

Procedure inverteren via $(A|\mathbb{I}_n)$: snappen hoe dit in mekaar zit. En ook onthouden: als je nulrij uitkomt, dan niet-inverteerbaar. Als je $\mathbb{I}_n)$ uitkomt, dan wel inverteerbaar. 

Vraag die ik me stelde: zijn volgende twee uitspraken equivalent? en geldt dit dan voor alle $B$? Antwoord is denk ik ja, zie verder bij Cramer op p.75. 
\begin{itemize}
    \item $A\times X = 0$ heeft enkel de triviale oplossing $X=\mathbb{O}$
    \item $A \times X = B$ heeft 1 oplossing 
\end{itemize}

LU decompositie. O.a. te onthouden: 
\begin{itemize}
    \item echelon en trapvorm zijn bovendriehoeks
    \item elemenataire matrices type I zijn boven- en benedendriehoeks; type II zijn geen van beide; type III zijn benedendriehoeks als $j<i$
\end{itemize}


% H2
\section{Determinanten}

% H3
\section{Vectorruimten}

% H4
\section{Lineaire afbeelingen en transformaties}

% H5
\section{Eigenwaarden, eigenvectoren, diagonaliseerbaarheid}


% H6 
\section{Inproductruimten en Euclidische ruimten}

Hoogtepunt cursus: p.273: spectraalstelling i.v.m symnmetrische of Hermitische matrices. 
Legt verband tussen: 
\begin{itemize}
    \item Euclidische ruimte (dus vectorruimte met inproduct)
    \item lineaire transformatie
    \item matrices
    \item orthonormale basis
    \item eigenvectoren
\end{itemize}

\end{document}