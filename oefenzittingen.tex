\documentclass{article}
\usepackage{amsmath}
\usepackage{amssymb}
\setlength{\parindent}{0em}


\title{Oefenzittingen lineaire algebra}
\author{Raf Meeusen}
\date{2023-2024}


\begin{document}
\maketitle

\section{Oefenzitting 1, 3 okt, week 2}

(10u30, lokaal 01.14; Tom geeft oefenzittingen). 

Oefeningen 1.7 pagina 47-54. 

Oefening 1 matrix A1 gemaakt. 
Oefening 2 matrix B1 gemaakt. 
Oefeningen 4a en 4b gemaakt. Denk wel juist. 
Oefening 5d gemaakt. 
Oefening 6d gemaakt, maar niet zo heel netjes opgeschreven. 
Oefening 10 linkse stelsel gemaakt. Denk redelijk juist gemaakt. 
Oefening 11b gemaakt. Wel juist in grote lijnen. 
Oefening 12 gemaakt. Maar niet goed aangepakt, en nog eens opnieuw gedaan. Kwam derdegraadvergelijking in b in, die je moet oplossen door nulpunt (b=1) te raden, en dan veelterm te delen. 



Oefening 14 begonnen. Maar gestopt, te veel schrijfwerk. 



\section{Oefenzitting 2, 10 okt, week 3}

\subsection{Voorbereiding}

Oefening 1.7-19 p.51 gemaakt. Deel 2 $C C^T$ kwam symmetrische matrix uit. Misschien altijd zo? Element $c_{ij}$ van product komt uit rij $i$ van linkse matrix en kolom $j$ van rechts. Element $c_{ji}$ komt uit rij $j$ van linkse en kolom $i$ van rechts. Rij $i$ van linkse matrix is kolom $i$ van rechtse. En kolom $j$ van linkse is rij $j$ van rechtse matrix. Dus inderdaad altijd zo. 


Oefening 1.7-20 p.51. Volgens mij is de clou dat na herschrijven volgt dat $A-A^T = -2B^T$, dus dat we hieruit makkelijk $-2B^T$ kunnen halen, en dan ook $B$. 

Oefening 1.7-24 matrix A1 p.52. OK, inverse berekend met truuk $(A|I)$. Uitkomst: 
\begin{math}
\begin{pmatrix}
-\frac{2}{7} & 1 & \frac{6}{7}\\
-\frac{1}{7} & 1 & \frac{3}{7} \\
\frac{4}{7} & -1 & -\frac{5}{7} 
\end{pmatrix}
\end{math}

Nagekeken met online calculator, klopte. Ook nog eens gekeken en nagedacht hoe te schrijven als product van elementaire. Niet vergeten dat het ook inverse van elementaire worden op een gegeven moment! Niet helemaal uitgeschreven, maar ik snap het principe wel. 

Oefening 1.7-34 p.53. Ok, gemaakt. Deel a) met inductie en ook uit ongerijmde (als k-de macht verschillend van nul is, is ook (k+1)-de macht verschillend van nul), en dan gezien dat dat niet kan. Deel b) gewoon uitgerekend. Die inverse die voorgesteld wordt kan altijd berekend worden, dan vermendigvuldigd met $I-A$ en uitgerekend.


\subsection{Oefenzitting}

Oefening 27(a) p.52: OK uitgerekend, klopte. 

Opdracht 1.27 op p.34: OK, klopt na uitschrijven van Tr(A-B) en dan te hergroeperen. 

Oefening 1.7-21 p.51
21a) $AA^T$: OK. Element $c_{ij}$ uitschrijven als som, dan meepakken dat $b_{kj} = a_{jk}$ voor de getransponeerde, en dan ook eens element $c_{ji}$ uitschrijven, en dan zie je dat ze hetzelfde zijn. Matrix met $c_{ij} = c{ji}$ is een symmetrische. Voor $A+A^T$ is zelfs nog makkelijker. 
Andere manier: definitie van symmetrisch: $A=A^T$. Dus als we $(A+A^T)^T$ uitrekenen en opnieuw $A+A^T$ uitkomen, dan is deze inderdaad symmetrisch. En er is een eigenschap van de getransponeerde van een som, dus kan ook makkelijke zo aangetoond worden. Ook voor product moet dat waarschijnlijk zo kunnen (eigenschap $(AB)^T$), maar niet verder uitgewerkt. 
22b) is ook redelijk triviaal. Dan 22c), gewoon uitgerekend dat som klopt. Dan 22d), was wat moeilijker. Gesteld dat $A=B+C$ met $B=B^T$ en  $C=-C^T$. Dan kan je berekenen dat die B en C gelijk zijn aan de uitdrukkingen uit vorige deeloefening door uitdrukkingen voor $A$ en $A^T$ verder uit te schrijven en op te lossen naar $B$ en $C$. 
21e geskipt. 


Oefening 1.7-35 p.53. Idempotente matrix. Ene vinden: eenheidsmatrix, en eerste rij maal -1 doen. Of tweede rij, etc. Deel b is makkelijk te berekenen. 

Oefening 1.7-39 p.54. 39a is vals. Tegenvoorbeeld is ooit in de les gegeven. 39b is waar. Uit gegeven volgt dat $A=A^{-1}$ en $B=B^{-1}$. Dan is $(AB)^{-1} = (A^{-1}B^{-1})^{-1}$. Via eigenschap inverse van product volgt het gevraagde. 
39c: vals. Stel $B=-A$, dan is $A+B=0$; tegenvoorbeeld.
39d: waar. 

39e: moeilijk, ik zie hem niet. Blijkbaar vals, moet paar keer proberen met bvb. 2x2 matrix. 

39f: makkelijk, ongerijmde. Stel AB inverse wel, dan moet A wel inverteerbaar zijn. 

39g: vals. Bvb. twee rijen wisselen, en een rij met  lambda vermenigvuldigen; operaties omkeren, en het klopt niet meer. 



\section{Oefenzitting 3, 17 okt, week 4}

\subsection {Voorbereiding}


Oefening 2.4-1 (p. 78):  
\begin{itemize}
\item Matrix A1: OK, berekend op 2 manieren (resultaat: 22) 
\item Matrix A6: OK: berekend op 2 manieren (resultaat: 12) 
\item Matrix A8: niet gemaakt 
\end{itemize}

Oefening 2.4-3 (p. 78):   
Gebruikmakend van $det(AB)=det(A)det(B)$: ofwel $det(A)=0$, ofwel $det(B)=0$. Uit eerste volgt: $x=4$ of $x=1$. Uit tweede: $x=0$ of $x=-2$. 

Oefening 2.4-4 (p. 78) : 
Matrix A1: twee rijwissels om van $A1$ terug tot $A$ te komen, dus tweemaal tekenverandering determinant, dus $detA_1=detA$. 


\subsection{Oefenzitting}

(niet geweest, was nog wat ziek) 

Oefening 1.7-38(a) (p. 54): LU decompositie doen; eerst wat theorie gelezen, want ik kende het niet goed genoeg (ook niet in les gezien hoe dat juist werkt). Dan oefening gemaakt, niet moeilijk. Principes begrepen (herleiden naar trapvorm via $E_i$ waarbij de $E_i$ zelf alleen benedendriehoeks zijn, en trapvorm zelf is een bovendriehoeks, (etc. etc.). Opgemerkt dat LU-decompositie ook voor mxn matrices gedefineerd is, waarbij dan L een vierkante (mxm) is, en U een mxn. Ook nog niet helemaal duidelijk: hoe is boven/benedendriehoeks juist gedefineerd voor niet-vierkante? Ook een LDU-decompositie bekeken: een diagonaalmatrix links-buitenbrengen van een willekeurige matrix is gewoon per rij delen door corresponderende getal, ook interessant te zien dat elke diagonaalmatrix een product is van meerdere elementaire matrices van het type dat $\lambda R_i$ doet. Elementaire matrices overigens altijd vierkant. Inverteerbare ook. Maar je kan links- of rechts-inverse vinden van een niet-vierkante! 


Oefening 2.4-12 (p. 79): lang op gezocht, niet gevonden. 

Oefening 2.4-28 (p. 81-82; c/d was huistaak) 
hier: (a,b,f,g,h,i,j,k,n)

a) niet waar ; bvb. $det(\mathbb{I}_n)=1$ vs. $det(\mathbb{I}_n + \mathbb{I}_n) = 2^n$

b) waar? ik dacht eerst niet waar; geprobeerd, en bleek voorbeeld pro: 
$\begin{pmatrix}1 & 0 \\1 & 2 \end{pmatrix} \begin{pmatrix}0 & 1 \\1 & 2 \end{pmatrix} = \begin{pmatrix}0 & 1 \\2 & 5 \end{pmatrix}$. En $ \begin{pmatrix}0 & 1 \\1 & 2 \end{pmatrix} \begin{pmatrix}1 & 0 \\1 & 2 \end{pmatrix} = \begin{pmatrix}1 & 2 \\3 & 4 \end{pmatrix}$.  Nog eens nagedacht. Uiteraard waar. Want $det(AB)=det(A) det(B) = det(B)  det(A)$

c) en d) : zie huistaak 

e) vals; 

f)  waar, dit matrix bestaat niet; Ongerijmde: stel er bestaat een 3x3 matrix $A$ zodat $A^2=-\mathbb{I}_n$, dan zou $det(A^2) = (-1)^3 = -1$. Maar $det(A^2) = det(A)det(A) = (det(A))^2$. Kan niet negatief zijn. 

g) waar; Ongerijmde; Stel $det(A) \neq 0$ en $A^k=0$. Dan is $det(A^k)=det(0)=0$. Maar $det(A^k)$ is ook gelijk aan $[det(A)]^k$. Als $[det(A)]^k =0$, dan kan uitgangspunt $det(A) \neq 0$ niet kloppen. 

h) vals ; met $k^n$ zijn dat je buitenbrengt. 

i) vals ; tegenvoorbeeld, neem bvb. $2\mathbb{I}_3$ en schrijf als som: $2\mathbb{I}_3 = \begin{pmatrix}1 & 0 &0\\0 & 1& 0\\0&0&1 \end{pmatrix} + \begin{pmatrix}1 & 0 &0\\0 & 1&0\\0&0&2 \end{pmatrix}$ ; we weten dat $det(2\mathbb{I}_3) = 8$; terwijl $det\begin{pmatrix}1 & 0 &0\\0 & 1& 0\\0&0&1 \end{pmatrix} = 1$ en $det\begin{pmatrix}1 & 0 &0\\0 & 1&0\\0&0&2 \end{pmatrix} = 2$

j) waar; matrix $uv^T$ is een nxn matrix, en $v^T$ is een enkele rij;  $uv^T$kan geschreven worden als rijen onder mekaar $\begin{pmatrix}R_1=u_1v^T\\R_2=u_2v^T\\...\\R_n=u_nv^T\end{pmatrix}$; deze heeft rijen die allemaal veelvouden zijn van elkaar, bvb. $R_2 = \frac{u_2}{u_1}R_1$, en dus kan hij herleid worden tot matrix met nulrij via een rij-operatie $R_x \rightarrow R_x+\lambda R_y$, en dus is de determinant $0$. 

k) waar; combinatie van twee stellingen: $AX=B$ heeft \'e\'en oplossing asa $A$ is inverteerbaar (stelling 1.39); $A$ is inverteerbaar asa $det(A) \neq 0$ (stelling 2.4). 

l) niet gemaakt

n)  niet gemaakt 


\section{Oefenzitting 4, 24 okt, , week 5}

\subsection {Voorbereiding}

Oefening 2.4-28(l) (p.82): waar. Definitie $adj(A) = det(A) A^{-1}$. Uit gegeven volgt dan hier: $adj(A) = A^{-1}$. Opnieuw $adj$ nemen van linkse en rechtse geeft: $adj(adj(A)) = adj(A^{-1})$. Maar $adj(A) = A^{-1}$ dus is $adj(A^{-1}) = {(A^{-1}})^{-1} = A$, en dus $adj(adj(A)) = A$. QED. 

Oefening 2.4-25 (p.81): oefening op Cramer. Truuk van Cramer is per onbekende een kolom vin $A$ vervangen door kolommatrix $B$, en dan determinant hiervan berekenen, en delen door $det(A)$, wat dan $x_i$ geeft met $i$ de kolom die is vervangen. Uitgerekend en nagekeken, klopte. 

Bewijs lemma 3.7 (p.95): Gegeven $v+x=w+x$. Te bewijzen: $v=w$. Bewijs: gelijkheid $v+x+x'=w+x+x'$ volgt uit gegeven, voor elke $x'$, en wegens associativiteit dan ook $v+(x+x')=w+(x+x')$ voor elke $x'$. Kiezen we $x'$ het tegengesteld element van $x$, dan wordt dit: $v+(0)=w+(0)$, en hieruit volgt $v=w$, want $0$ is het NE. 

\subsection {Oefenzitting}

Oefening 2.4.24 (p. 81): $A$ en $B$ inverteerbaar. 

\begin{enumerate}
\item[(a)] Toon aan dat $adj(AB)=adj(B) \cdot adj(A)$. Bewijs: we hebben eigenschappen $A \cdot adj(A) = det(A) \mathbb{I}_n$, $A^{-1} = \frac{1}{det(A)} adj(A)$ en $adj(A) = det(A)A^{-1}$. Eerste eigenschap toepassen op $AB$: $AB \cdot adj(AB) = det(AB) \mathbb{I}_n$. Links en rechts links-vermenigvuldigen met $(AB)^{-1}$: $adj(AB) = (AB)^{-1} det(A) det(B) = B^{-1} A^{-1} det(A) det(B) =  det(B)B^{-1} det(A)A^{-1}$. Hieruit volgt het te bewijzen via derde eigenschap. 

\item[(b)] $adj(QAQ^{-1}) = adj(Q^{-1}) adj(A) adj(Q) $ via vorige eigenschap in a). Dan $adj(Q)=det(Q) Q^{-1}$ toepassen op $Q$ en ook op $Q^{-1}$, en wetende dat $det(X^{-1}) = \frac{1}{det(X)}$, geeft het gevraagde. 

\item[(c)] Gegeven $AB=BA$. Hint gekregen in oefeningzitting: gebruik vorige eigenschap. 
$AB=BA$, dan is $ ABB^{-1} = BAB^{-1} $ of $A = BAB^{-1}$. Dan is ook  $adj(A) = adj(BAB^{-1})$. Voor rechterlid kan eigenschap uit (b) gebruikt worden, en dan is: $adj(A) = B adj(A) B^{-1}$. Links en rechts via rechter-matrix-vermenigvuldiging met $B$: $ adj(A) B = B adj(A) $. QED. 



\end{enumerate}

Oefening 2.4.29 (p. 82-83): begonnen, maar best moeilijk. Hint in opgave suggereert bewijs door volledige inductie. Niet meer zelf gemaakt, maar in oefenzitting heeft Tom drie verschillende bewijzen (of toch outlines van bewijzen) gegeven. 
\begin{enumerate}
    \item[(1)] bewijs alternatief 1: via volledige inductie. 
    \item[(2)] bewijs alternatief 2: via determinant van een trapvorm (stelling 2.4, driehoeksmatrix: product diagonaalelementen), en via een Lemma dat $det(P)=det(A)det(B)$ blijft gelden na een ERO. 
    \item[(3)] bewijs alternatief 3: via formule (2.2) voor determinanten. 
\end{enumerate}


Bewijs alternatief 1: stap 1: we tonen aan dat $det(P)=det(A).det(B)$ geldig is voor $k=1$. Gewoon ontwikkelen naar 1e kolom, geeft $det(P)=a_{11}det(B)= det(A)det(B)$. QED. 
Dan stap 2, de inductiestap. Inductiehypothese: we nemen aan dat voor $A \in \mathbb{R}^{(k-1)\times (k-1)}$ geldt: $det(P) = det(A)det(B)$. Schrijven we dan $det(P)$ uit voor $A \in \mathbb{R}^{k\times k}$: 
\[ det(P) = det 
\begin{pmatrix}
a_{11} & ... & a_{1k}  &  \\
       & ... &   &  C \\
a_{k1} & ... & a_{kk}  & \\
       &   &         &   \\ 
       & O   &         & B 
\end{pmatrix} \]  
en ontwikkelen we naar kolom 1 ($i = $ rij-index, en wetende dat we nultermen hebben voor $i > k$):  
\[ det(P) = \sum_{i=1}^k (-1)^{i+1} a_{i1} det(M_{i1})  \]
met 
\[ det(M_{i1}) = det \begin{pmatrix}
a_{12} & ... & a_{1k}  &  \\
       & ... &   &   \\
a_{(i-1)2} & ... & a_{(i-1)k}  & C_i\\
a_{(i+1)2} & ... & a_{(i+1)k}  & \\
       & ... &   &   \\
a_{k2} & ... & a_{kk}  & \\
       &   &         &   \\ 
       & O   &         & B 
\end{pmatrix} = det \begin{pmatrix}
A_i  &  C_i \\
 O   &  B 
\end{pmatrix} \]
waarbij $A_i$ een $(k-1) \times (k-1)$ matrix is omdat de eerste kolom van $A$ is weggelaten, en de $i$-de rij van $A$ geschrapt is (en $C_i$ eveneens van $C$ is afgeleid door de $i$-de rij te schrappen). 
Nu is $M_{i1}$ een matrix die voldoet aan de vorm van matrix $P$ van de inductiehypothese, dus $det(M_{i1}) = det(A_j) det(B) $. Dus geldt: 
\[ det(P) = \sum_{i=1}^k (-1)^{i+1} a_{i1} det(A_j) det(B) =  det(B) \sum_{i=1}^k (-1)^{i+1} a_{i1} det(A_i) \]
Hierin is de som nu gewoon de ontwikkeling van de originele matrix $A$ naar de eerste kolom, dus staat er: 
\[ det(P) = det(B)det(A) = det(A)det(B)\]
QED. 

Bewijs alternatief 2: niet uitgewerkt, eventueel nog een TODO. 

Bewijs alternatief 3: heeft op bord gestaan, met truuk door sommatie over $\sigma \in \S_n$ te splitsen over twee sommaties $\sigma_1 \in S_k$ en $\sigma_2 \in S_{n-k}$. Niet helemaal gesnapt, maar denk wel dat ik die zou kunnen vinden als ik wat zoek. 


Oefening 3: zie opgave (vrij lang). Even naar gekeken, en op papier wat aan gewerkt. Zag er tegenop om me bezig te houden met die speciale + en speciale $\cdot$ notatie. Leek me allemaal wel triviaal, maar gewoon heel lastig om telkens uit te schrijven dat $\forall x: f(x)=...$. 

Oefening 4 = Doe opdracht 3.9 = p. 96 in boek. Toon aan dat $\lambda v = 0 \implies \lambda = 0$ of $v=0$. Contrapositie, we mogen ook aantonen $\lambda \neq 0 \land v \neq 0 \implies  \lambda v \neq 0$ . Uit definitie 3.3, co\"eff.1: $1.v = v$. Omdat $\lambda \neq 0$, volgt $ (\lambda \frac{1}{\lambda}) v = v$, of ook $  \frac{1}{\lambda} ( \lambda v ) = v$. Maar $v \neq 0$, dus is ook $  \frac{1}{\lambda} ( \lambda v ) \neq 0 $. In ongelijkheid links en rechts maal $\lambda$ geeft: $ \lambda v \neq 0$ (via Lemma 3.7 $\lambda 0 = 0$). 

Oefening 5: Maak oefening 3.6.3. (p. 128): Wat neergesschreven en geprobeerd: optelling niet commutatief. Ik vind geen neutraal element. Optelling niet associatief. Gemengde associativiteit wel OK.  



\section{Oefenzitting week 6}

\subsection{Voorbereiding}
Voor te bereiden: 3.6: 
oef. 2(W2, W5, W6, W8), 
oef. 6
Opdracht 3.25

\subsection{Oefenzitting}
Zie pdf met opgaven. 

\end{document}