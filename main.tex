\documentclass{article}
\usepackage{graphicx} % Required for inserting images
\usepackage{amssymb}

\title{Nota's lineaire algebra}
\author{Raf Meeusen}
\date{2023-2024}

% mark paragraphs with empty line instead of indented first line
\setlength{\parindent}{0em}
\setlength{\parskip}{1em}

\begin{document}

\maketitle

\section{Les 1, 26 sept}

Toepassingen matrices: 
\begin{itemize}
\item google search (eigenwaarden/eigenvectoren worden berekend) 
\item jpeg compressie (projecties uit lin. alg.) 
\end{itemize}


Lineair: zelfs voor niet-lineaire problemen wordt vaak een lineaire benadering (ook: eerste orde benadering) gebruikt voor een oplossing; bvb: slinger waarbij sin theta = theta wordt gesteld

Examen: 
\begin{itemize}
\item duurt 3 uur
\item gesloten boek
\item schriftelijk
\item 2 oefeningen + 1 theorievraag
\item er zijn voorbeeldexamens 
\end{itemize}

Boek: editie 2 ook bruikbaar ; maar er is een editie 3 


Makkelijke verbanden tussen lin. alg. en andere topics: 
\begin{itemize}
    \item fysisch netwerk met knooppunten (elek, stroom auto's, vloeistof): knooppuntvergelijkingen geven stelsel; stelsel = lin. algebra
    \item vlakke meetkunde: 2 vgln. stellen rechte voor; stelsel = alle punten die op beide rechten liggen
    \item Ruimte-meetkunde: vergelijkbaar, vlakken / rechten
\end{itemize}

Vrijheidsgraden als er oneindig veel oplossingen zijn. Oplossingen te schrijven in functie van "vrije parameters". 

"operaties die de oplossingen van een stelsel bewaren" 
(zie ook oefening nabeschouwing met S en S') 
Deze operaties zijn per definitie omkeerbaar.  

Termen: 
\begin{itemize}
\item uitgebreide matrix van een stelsel (met de $b_i$) 
\item achterwaartse subsitutie (bij oplossen stelsel) 
\end{itemize}

Stelsels waarbij oplossing bijna rechtstreeks af te lezen is:  voorbeelden gezien in de les. Allemaal met een matrix die in specifieke vorm is. Dan link gelegd met echelonvorm / trapvorm. 

Termen:
\begin{itemize}
\item echelonvorm
\item trapvorm
\item gebonden variabelen (kolommen met leidende 1) 
\item vrije variabelen (zonder leidende 1); als parameter te kiezen
\item strijdig stelsel
\end{itemize}

Gewoonte: parameter $\lambda$, $\mu$, ...

Elk stelsel kan naar echelon geconverteerd worden. MAAR: een echelonvorm is NIET uniek. 

Methode: algoritme, technieken. 

Dan nog gezien: stelling 1.11. (ivm. 1 oplossing, 0 oplossingen, oneindig veel oplossingen)

Notatie: $\mathbb{R}^{nxk}$ , voor n rijen / k kolommen. 

Voorbeelden bewijzen eigenschappen matrices: onthouden dat een matrix defini{\"e}ren erop neerkomt om elk element te bepalen (notatie Aij = element op rij i en kolom j). 

Definitie: transponeren v/e matrix ($A^T$) . Is spiegelen over diagonaal. 

Termen:
\begin{itemize}
\item vierkante matrix
\item symmetrische matrix ($A = A^T$) ; moet dus vierkant zijn
\item diagonaalmatrix
\end{itemize}

\section{Les 2, 3 okt}


Matrices vermenigvuldigen. Te onthouden: rij links, kolom rechts. Een rij x een kolom = 1 getal. (makkelijk te onthouden, want ik weet ook nog dat nxm matrix maal mxp matrix een nxp matrix geeft, en nxm matrix heeft n rijen met lengte m, en de lengte van de kolommen van de tweede is ook m. Logisch. Aantal rijen = lengte van de kolommen, en vice versa. 

% even vooruitlopen waarom dat matrixproduct zo gedefinieerd is: 
% 
%Functie $L_A(x)$ is een afbeelding van $\mathbf{R}$ $$ R^k naar R^n, met 
%L_A(x) = som (Aij . xj) 
%(elke matrix A definieert een afbeelding La van ...) 
%En dan is La . Lb = L_A.B 
%(en dit maakt de definitie van matrixproduct logisch) 


\begin{itemize}
    \item veel eigenschappen en rekenregels
    \item een nulmatrix $O_{nxk}$
    \item een eenheidsmatrix $I_n$ (vierkante matrix) 
    \item MAAR: vermenigvuldiging niet-commutatief (logisch want niet eens gedefinieerd voor alle nxk en kxl dimensies. 
    \item $AB$ kan nul zijn terwijl $A \neq O$ en $B \neq O$
\end{itemize}


Noot: als we een matrix zien als een "operatie", en een vermenigvuldiging van matrices zien als de opeenvolging van die twee operaties, dan vinden we het ook logisch dat de uitkomst van $AB$ niet zelfde is als $BA$. (schoenen aantrekken, dan kousen, of omgekeerd. 

Matrices waar volgorde van operaties wel omwisselbaar is: noemt men commuterende matrices. 

Vraag: kunnen we matrices delen? Aangezien product van matrices nul kan zijn zonder dat een van de matrices nul is, is het delen niet zo evident als bij re{\"e}le getallen. Bvb. uit $ab=0$ volgt, als $b=0$, dat $a=0$, omdat we mogen delen door $b$. 

Inverteren: een matrix vinden zodanig dat vermenigvuldiging de eenheidsmatrix geeft. Ook hier zijn er verschillende opties: de linkerinvers en de rechterinvers. 

Opmerking: we beperken ons hier tot vierkante matrices wat betreft inverse, hoewel je ook inverse kan berekenen voor niet-vierkante matrices. 

Stelling: voor een vierkante matrix die links inverse B heeft en rechts inverse C, geldt dat $B=C$. (bewijs in les gezien, heel kort). 

Definitie 1.33. We noemen een nxn matrix ofwel inverteerbaar, ofwel niet-inverteerbaar, indien ... Synoniemen: regulier, niet-singulier/singulier. Opgelet: op dit moment hebben we nog niet bewezen dat linker inverse en rechter inverse gelijk zijn! Dat komt later! 

Stelling 1.34: inverse van product (als beide inverteerbaar zijn) =  product van de inversen in omgekeerde volgorde. Het bewijs is gebaseerd op: aantonen dat een gegeven matrix een inverse van een andere, is bewijzen dat zowel links-vermenigvuldigen als rechts-vermenigvuldigen resulteert in de eenheidsmatrix. 

Noot: niet evident hier om de te volgen wat al bewezen is, en wat we al weten, maar nog niet bewezen is (en dus: wat volgt eigenlijk waaruit). Bvb. nog steeds niet bewezen dat linker en rechter inverse gelijk gaan zijn. 

Rap een eigenschap tussendoor: als $AB=0$ en $A$ is inverteerbaar, dan is $B=0$. (bewijs redelijk makkelijk) 

Term/concept: elementaire matrix $E_n$. Komt voort uit oplossen van stelsels met matrix rij-operaties, gecombineerd met matrix-vermenigvuldigen. We kunnen namelijk rij-operaties voorstellen als matrix-vermenigvuldiging met zogenaamde elementaire matrices. 

Interessant: twee rijen van plaats wisselen: ongedaan maken is opnieuw hetzelfde doen! Dus inverse van $E_n$ voor rijen verwisselen, is gewoon dezelfde $E_n$! 

Noteer ook: alle elementaire matrices zijn inverteerbaar. 

ERO: elementaire rij-operaties. Bij elke ERO hoort een EM (elementaire matrix), noemen we $\Sigma$, en die EM bekom je door rij-operaties toepassen op de eenheidsmatrix. ERO toepassen op matrix A komt overeen met matrix-vermenigvuldiging $\Sigma A$. Links-vermenigvuldigen dus. 

Noot: we weten dat twee rijen wisselen, ongedaan gemaakt wordt door opnieuw hetzelfde te doen. Dus: de matrix $\Sigma$ van rij-wissel operatie, moet wel de inverse van zichzelf zijn. 

Grote stelling: 1.39, met 6 equivalente beweringen. De matrix A heeft een links inverse B; het stelsel $AX=0$ heeft enkel de evidente oplossing $X=0$ (dus stelsel met 1 oplossing!); etc. etc. Puntje 4: matrix A is product van elementaire matrices, is een soort sleutel-formulering, en ook een sleutel in het grote bewijs. 

Bewijs: formulering 2 volgt uit 1 door links met B te vermenigvuldigen en uit te werken. Formulering 3 volgt uit 2, doordat stelsels met 1 oplossing in rij-echelon-vorm de eenheidmatrix hebben, en altijd daartoe te herleiden zijn met ERO's. Bewijs van 4 uit 3 (A is product van EM's): op bord geschreven tijdens de les. Ook redelijk makkelijk (moet wel weten dat EM's inverteerbaar zijn, en opnieuw EM geven). 

Tussenstelling: als $C$ inverteerbaar, en $CA=I_n$, dan is A inverteerbaar en $A^{-1}=C$. Het bewijs: als $CA=I_n$, dan is A rechterinvers van C, en dus $A=C^{-1}$

Noot: tijdens de les effe de draad kwijt tussen wat al bewezen was, en wat niet, en de definities inverteerbaar etc. Nog eens nakijken! 

Noot2: belangrijk is: als $AB=I_n$, dan kan je dat op 2 manieren lezen: A is linkerinvers van B, of B is rechterinvers van A. Die twee manieren van lezen heb je nodig om bewijs sluitend te krijgen. 

Bovendien volgt uit deze stelling een algoritme om de inverse te berekenen: doe dezelfde rijoperaties om A te rij-herleiden naar $I_n$, ook gewoon telkens op $I_n$, bijvoorbeeld door de matrix $(A|I_n)$ te gebruiken voor alle rij-operaties. Dan krijgen we rechts $A^{-1}$, want in bewijs hadden we gezien dat $A^{-1} = \Sigma_k \Sigma_{k-1} ... \Sigma_1$. Als bij A een nul-rij ontstaat, is A niet inverteerbaar. 

Dan nog even over LU-decompositie gehad. (lower / upper triangular = benedendriehoeksmatrix en bovendriehoeksmatrix). Benedendriehoeksmatrix heeft niet-nullen op en beneden diagonaal. Het kunnen schrijven van een matrix A als product van een L en een U, heeft interessante toepassingen. Ook interessant: product van twee bovendriehoekmatrices is opnieuw bovendriehoeks. Een diagonaalmatrix is zowel bovendriehoeks als benedendriehoeks. Kijk ook eens naar de EM's en of ze L of U zijn. Echelonvorm is bovendriehoeks. 

Ook "de spil" vernoemd hier ivm rij-herleiden. Dus ook leerstof (spil/gauss-eliminatie...). En volgende redenering: als je A in echelonvorm kan zetten zonder rijen te verwisselen, en we weten dat echelonvorm een U-vorm is. Dus $U= (\Sigma_k ... \Sigma_1) A$, met $\Sigma_k ... \Sigma_1$ inverteerbaar, en inverse hiervan is een L-vorm (indien geen rij-wissels). 

\section{Les 2 nabeschouwing}

1. stelsel herleiden naar trapvorm, dan elementaire matrices hiervoor opschrijven, en dan A herschrijven als product van elementaire (die dan inverse zijn van oorspronkelijke elementaire!) en de trapvorm zelf. 
OK, alleen twijfelde ik wat bij inverteren van de elementaire. Niet slecht om eens voorbeeld van op te schrijven (1/lambda, en +lamba Ri wordt -lambda Ri). 

2. stelling 1.38 bewijzen; had ik al gedaan voor paar getallenvoorbeelden, moet in deze oefening eigenlijk gewoon met variabele lambda en zo, dus heb ik geskipt. Wel goeie oefening om die inversen van elementairen eens uit te schrijven. 

3. Die bewijzen zijn veel schrijfwerk, en met al die verschillende indices en mxn, nxp etc. dimensies, ... Eerst eens proberen definitie 1.22 zelf op te schrijven zonder te spieken. 

4. Doenbaar, wat rekenwerk. Gevolg 1.40 is pagina 40. Maar nog niet gedaan. 

5. Oefening 1.35 uit boek (is op p. 53, idempotente matrix). Nog niet gedaan. 

\section{Les 3 voorbereiding}



\section{KEEP ME TO JUMP TO END}

\end{document}
