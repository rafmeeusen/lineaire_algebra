\documentclass{article}
\usepackage{graphicx} % Required for inserting images

\title{Nota's lineaire algebra}
\author{Raf Meeusen}
\date{2023-2024}

\begin{document}

\maketitle

\section{Les 1, 26 sept}

Toepassingen matrices: 
\begin{itemize}
\item google search (eigenwaarden/eigenvectoren worden berekend) 
\item jpeg compressie (projecties uit lin. alg.) 
\end{itemize}


Lineair: zelfs voor niet-lineaire problemen wordt vaak een lineaire benadering (ook: eerste orde benadering) gebruikt voor een oplossing; bvb: slinger waarbij sin theta = theta wordt gesteld

Examen: 
\begin{itemize}
\item duurt 3 uur
\item gesloten boek
\item schriftelijk
\item 2 oefeningen + 1 theorievraag
\item er zijn voorbeeldexamens 
\end{itemize}



Boek: editie 2 ook bruikbaar ; maar er is een editie 3 

Makkelijke verbanden tussen lin. alg. en andere topics: 
fysisch netwerk met knooppunten (elek, stroom auto's, vloeistof): knooppuntvergelijkingen geven stelsel; stelsel = lin. algebra
vlakke meetkunde: 2 vgln. stellen rechte voor; stelsel = alle punten die op beide rechten liggen
Ruimte-meetkunde: vergelijkbaar, vlakken / rechten
Vrijheidsgraden als er oneindig veel oplossingen zijn. Oplossingen te schrijven in functie van "vrije parameters". 

"operaties die de oplossingen van een stelsel bewaren" 
(zie ook oefening nabeschouwing met S en S') 
Deze operaties zijn per definities omkeerbaar.  

Term: uitgebreide matrix van een stelsel (met de bi) 
Term: achterwaartse subsitutie (bij oplossen stelsel) 

Stelsels waarbij oplossing bijna rechtstreeks af te lezen is => voorbeelden => allemaal met een matrix die in specifieke vorm is => link gelegd met echelonvorm / trapvorm. 

Term: echelonvorm
Term: trapvorm

Term: gebonden variabelen
Term: vrije variabelen
(af te lezen in echelonvorm: kolommen met leidende 1 = gebonden variabelen; zonder leidende 1 = vrije variabelen = als parameter kiezen)

Gewoonte: parameter lambda, mu, ... 

Term: strijdig stelsel. 

Elk stelsel kan naar echelon geconverteerd worden. 
MAAR: een echelonvorm is NIET uniek. 

Methode: algoritme, technieken. 

Dan nog gezien: stelling 1.11. (ivm. 1 oplossing, 0 oplossingen, oneindig veel oplossingen)

Notaties Rnxk  : n rijen / k kolommen. 

Voorbeelden bewijzen eigenschappen matrices: onthouden dat een matrix definieren erop neerkomt om elk element te bepalen (notatie Aij = element op rij i en kolom j). 

Definitie: transponeren v/e matrix (AT) . Is spiegelen over diagonaal. 

Termen: 
vierkante matrix
symmetrische matrix (A = AT) ; moet dus vierkant zijn 
diagonaalmatrix



\section{Les 2, 3 okt}

\end{document}
