\documentclass{article}
\usepackage{amsmath}
\usepackage{amssymb}
\setlength{\parindent}{0em}
\setlength{\parskip}{1em}

\title{Huistaak 1 lineaire algebra}
\author{Raf Meeusen}
\date{week 4}

\begin{document}
\maketitle

\section{Oefening 2.4-16, p.80}

Gegeven: $A^T=-A$, $A \in \mathbb{R}^{nxn}$, n oneven. 

Te bewijzen: $det(A) = 0$

Bewijs: aangezien $A^T=-A$, geldt dat $det(A^T)=det(-A)$. Uit stelling 2.4: $det(A^T)=det(A)$. 
Dus: $det(A)=det(-A)$. Wegens rijlineariteit van determinant in elke rij: $det(-A) = (-1)^ndet(A)$. 
Dus voor oneven $n$ geldt dat $det(-A) = -det(A)$, wat alleen kan als $det(A)=0$. 

Voorbeeld A met even n (n=4) waarbij $det(A) \neq 0$: 
\begin{math}
\begin{pmatrix}
0 & 0 & 0 & -1 \\
0 & 0 & -1 & 0 \\
0 & 1 & 0 & 0 \\
1 & 0 & 0 & 0 \\
\end{pmatrix}
\end{math}
waarbij $det(A) = 1 $

\section{Oefening 2.4-28}

Huistaak: deel c) en deel d) 
 
c) niet waar. Bvb. voor $c=0$: $det(-A) = (-1)^n detA$

d) waar; we weten dat $det(A^T) = det(A)$ (eigenschap stelling 2.4); dus laat ons kijken of we iets interessant vinden als we $(c\mathbb{I}_n - A^T)^T$ berekenen: $(c\mathbb{I}_n - A^T)^T = (c\mathbb{I}_n^T - {A^T}^T) = (c\mathbb{I}_n - A)$, wegens: transponeren van som van matrices is som van getransponeerden, en getransponeerde van een diagonaalmatrix geeft zelfde matrix terug. Hieruit volgt de gestelde bewerking, QED.  

\end{document}